\documentclass[a4paper]{article}
\usepackage{url}
\usepackage[margin=2cm]{geometry}
\usepackage{amsmath}
\title{governing-equations-of-the-rcm}

% \newcommand{\med}[1]{[#1]^m}
\newcommand{\med}[1]{M#1}
% \newcommand{\cell}[1]{[#1]^c}
\newcommand{\cell}[1]{C#1}


\newcommand{\MA}{\med{A}}
\newcommand{\MB}{\med{B}}
\newcommand{\MBH}{\med{BH}}
\newcommand{\MGluca}{M_{glucamine}}
\newcommand{\MNa}{\med{Na}}
\newcommand{\MK}{\med{K}}
\newcommand{\MCatp}{\med{Ca^{2+}}}
\newcommand{\MMgtp}{\med{Mg^{2+}}}
\newcommand{\MGluco}{M_{gluconate}}
\newcommand{\MH}{\med{H}}
\newcommand{\MOs}{\med{Os}}
\newcommand{\MCat}{\med{Cat}}
\newcommand{\MMgt}{\med{Mgt}}


\newcommand{\KB}{K_B}
\newcommand{\KBCa}{K_{{BC}_a}}

\newcommand{\CNa}{\cell{Na}}
\newcommand{\CK}{\cell{K}}
\newcommand{\CH}{\cell{H}}
\newcommand{\CMgtp}{\cell{Mg^{2+}}}
\newcommand{\CCatp}{\cell{Ca^{2+}}}
\newcommand{\CA}{\cell{A}}
\newcommand{\nHb}{n_{Hb}}
\newcommand{\CHb}{\cell{Hb}}
\newcommand{\nX}{n_{X}}
\newcommand{\CXm}{\cell{X^{-}}}
\newcommand{\COs}{\cell{Os}}
\newcommand{\CCa}{\cell{Ca}}
\newcommand{\CCaB}{\cell{CaB}}
\newcommand{\CBCa}{\cell{B_{Ca}}}
\newcommand{\CMgB}{\cell{MgB}}
\newcommand{\CMg}{\cell{Mg}}

\newcommand{\QX}{QX}
\newcommand{\QCa}{QCa}

\newcommand{\pH}[1]{pH_{#1}}
\newcommand{\pI}{pI}

\newcommand{\fHb}{f_{Hb}}

\begin{document}
\maketitle

\section{Introduction}\label{introduction}

Red blood cell homeostasis addresses the subset of mechanisms that
control the dynamic changes in cell volume, membrane potential, ionic
composition, membrane transport and osmotic gradients in response to
perturbations. The modelled system consists of a suspension of identical
RBCs whose dynamic behaviour is constrained only by charge and mass
conservation. The equations implement these laws following a strict
computational sequence representative of the multiple interconnected
processes involved, delivering true and tested predictions on the
homeostatic behaviour of human RBCs in physiological, pathological and
experimental conditions.

We retain in the equations the names of parameters and variables as used for model inputs and outputs where the use of conventional nomenclatures for concentrations (i.e. [Ca2+]c, [Mg2+]o) or of Greek symbols for fluxes was not feasible.  Coherence with model operation in the choice of equation nomenclature was considered more important than compliance with established nomenclature, clarity of meaning being preserved in the unconventional style and fully explained in the Appendix glossaries of the User Guide (\url{https://github.com/sdrogers/redcellmodeljava}).


\subsection{The initial Reference State(RS)}

The RBC reference state describes the initial condition of the system in
a pump-leak balanced steady state. For compliance with initial
electroneutrality and osmotic equilibrium we use a phenomenology in
which the charge, $\nX$, and cell content of the global, non-haemoglobin,
impermeant cell anion, $\QX$ , are treated as wildcard parameters in
equations 3 and 8. The $\nX$ and $\CXm$ values emerging from such treatment
correspond closely with the known organic and inorganic phosphate pools
of metabolically normal RBCs {[}3{]}. When modifying the initial default
values in the RS the wildcard parameters may change. The model
automatically recalculates their value potentially changing slightly the
constitutive make up of the impermeant cell anion ($\nX \times \QX$) in
the new cell.

\subsubsection{Mediumelectroneutrality}\label{medium-electroneutrality}

\begin{equation}
\MA + (\MB - \MBH) + \MGluco-(\MNa + \MK + 2(\MCatp + \MMgtp + \MGluca)) = 0 \tag{1}
\end{equation}

Medium concentration of proton-bound buffer, $\MBH$ (HEPES, by default):

\begin{equation}
\MBH = \MB\left(\frac{\MH}{\KB + \MH}\right)
\end{equation}

\subsubsection{Intracellular electroneutrality}

\begin{equation}
\CNa + \CK + \CH + 2\CMgtp + 2\CCatp - (\CA + \nHb\CHb + \nX\CXm) = 0
\end{equation}

$\nHb$, the net charge on the haemoglobin molecule, is represented by the Cass-Dalmark equation [4],

\begin{equation}
\nHb=  a(\pH{i} - \pI)
\end{equation}

where $a$ corresponds to the linear segment of the proton titration curve of Hb in intact RBCs, and $\pI$ is the $\pH{i}$ at the isoelectric point of haemoglobin.

\subsubsection{Medium and cell osmolarities, MOs and COs:}

\begin{eqnarray}
	\MOs &=& \MA + \MB + \MGluco + \MNa + \MK + \MCat + \MMgt + \MGluca\\
	\COs &=& \CNa + \CK + \CA + \CH + \CMgtp + \CCatp  + \fHb\CHb + \CXm
\end{eqnarray}

$\fHb$ is the osmotic coefficient of haemoglobin, represented with only two virial coefficients, $b$ and $c$:

\begin{equation}
\fHb = 1 + b\CHb + c\CHb^2 
\end{equation}

\subsubsection{Osmotic equilibrium in the reference steady state:}

\begin{equation}
\MOs = \COs
\end{equation}

\subsubsection{Cytoplasmic buffering of protons, calcium and magnesium.}
Heamoglobin is the major cytoplasmic buffer for protons (eq 4) and for calcium ($\alpha$-buffer in eq 9c).  The main magnesium buffers are ATP and 2,3-DPG, compounds integrated within the X  phenomenology.  Because the bound forms of Ca and Mg are contained within CX , they are not included as separate osmolarity contributors in eq 6, leaving only the free forms of $Ca^{2+}$ and $Mg^{2+}$ as osmotic contributors.  

Cytoplamic $Ca^{2+}$ and $Mg^{2+}$ buffering have been measured with precision in intact RBCs [5-8] enabling accurate representations in the model.  The total Ca and Mg content of the cells, QCa and QMg, is reported in units of mmol/(340g Hb) (or mmol/Loc) whereas concentrations of the free forms, $\CCatp$ and $\CMgtp$, are expressed in units of mmol/Lcw, a conversion requiring translation for operational reasons in the model   Equation 9a translates QCa in units of mmol/Loc to CCa in units of mmol/Lcw using:


\setcounter{equation}{0}
\renewcommand{\theequation}{9.\alph{equation}}

\begin{equation}
\CCa = \QCa\frac{RCV}{Vw}  
\end{equation}
 

The total calcium concentration is the sum of free and bound forms:

\begin{equation}
\CCa = \CCatp + \CCaB
\end{equation}

There are two buffer systems for binding calcium in the RBC cytoplasm, $\alpha$ (mostly haemoglobin), and the BCa/KBCa buffer [6]. The concentration of bound calcium, CCaB, at each total calcium concentration, CCa, is represented by:

\begin{equation}
\CCaB = \alpha\CCa + \CBCa\frac{\CCatp}{\CCatp + \KBCa}
\end{equation}

CCa2+ is solved from the implicit equation: 

\begin{equation}
\CCa – \CCatp – \CCaB = 0  
\end{equation}

by the Newton-Raphson routine in the RS and at the end of the computations in each iteration cycle.  The measured values of the calcium binding parameters are $\alpha = 0.30$, $\CBCa = 0.026$ mmol/Loc, and $\KBCa = 0.014$ mM [6].

The corresponding equations for cytoplasmic magnesium buffering and $CMg^{2+}$ are:

\begin{eqnarray}
\CMg &=& QMg\frac{RCV}{Vw}\\
\CMg &=& \CMgtp + \CMgB\\
\CMgB &=& CBMg1\frac{\CMgtp}{\CMgtp+KBMg1}+CBMg2\frac{\CMgtp}{\CMgtp+KBMg2}+CBMg3
\end{eqnarray}

\emph{is $KBMG2$ actually $K \times BMg1$}

$\CMgtp$ is solved from the implicit equation: 

\begin{equation}
\CMg - \CMgtp - \CMgB = 0
\end{equation}


The measured values of the Mg bufferes [8] are: CBMg1 = 1.2 mmol/Loc, KBMg1 = 0.08 mM; CBMg2 = 7.5 mmol/Loc (15 mEq/Loc), KBMg2 = 3.6 mM; CBMg3 = 0.05 mmol/Loc. BMg1 represents ATP, BMg2 represents 2,3-DPG and miscellaneous phosphate groups, and BMg3 is an unidentified high affinity magnesium buffer.  

\setcounter{equation}{9}
\renewcommand{\theequation}{\arabic{equation}}
\begin{equation}
test
\end{equation}

\end{document}
